\documentclass[]{article}
\usepackage{amssymb}
\usepackage{amsmath}
\usepackage[utf8]{inputenc}
\usepackage{graphicx}
\usepackage{booktabs}
\usepackage{listings}
\usepackage{color}
\usepackage{tabularx}
\usepackage{hyperref}

\definecolor{dkgreen}{rgb}{0,0.6,0}
\definecolor{gray}{rgb}{0.5,0.5,0.5}
\definecolor{mauve}{rgb}{0.58,0,0.82}

\lstset{frame=tb,
	%language=C++,
	aboveskip=3mm,
	belowskip=3mm,
	showstringspaces=false,
	columns=flexible,
	basicstyle={\small\ttfamily},
	numbers=none,
	numberstyle=\tiny\color{gray},
	keywordstyle=\color{blue},
	commentstyle=\color{dkgreen},
	stringstyle=\color{mauve},
	breaklines=false,
	breakatwhitespace=true,
	tabsize=2
}


\title{FYS-STK4155 H20 - Project 2:\\Neural nets}
\author{Olav Fønstelien}

\begin{document}
\maketitle

\begin{abstract}
%The abstract gives the reader a quick overview of what has been done and the most important results. Try to be to the point and state your main findings. It could be structured as follows 
% - Short introduction to topic and why its important 
% - Introduce a challenge or unresolved issue with the topic (that you will try to solve) 
% - What have you done to solve this 
% - Main Results 
% - The implications

\end{abstract}

\section{Introduction} \label{intro}
%When you write the introduction you could focus on the following aspects
% - Motivate the reader, the first part of the introduction gives always a motivation and tries to give the overarching ideas
% - What I have done
% - The structure of the report, how it is organized etc


\clearpage
\section{Methods} \label{methods}
% - Describe the methods and algorithms
% - You need to explain how you implemented the methods and also say something about the structure of your algorithm and present some parts of your code
% - You should plug in some calculations to demonstrate your code, such as selected runs used to validate and verify your results. The latter is extremely important!! A reader needs to understand that your code reproduces selected benchmarks and reproduces previous results, either numerical and/or well-known closed form expressions.

\clearpage
\section{Results} \label{results}
% - Present your results
% - Give a critical discussion of your work and place it in the correct context.
% - Relate your work to other calculations/studies
% - An eventual reader should be able to reproduce your calculations if she/he wants to do so. All input variables should be properly explained.
% - Make sure that figures and tables should contain enough information in their captions, axis labels etc so that an eventual reader can gain a first impression of your work by studying figures and tables only.


\clearpage
\section{Discussion and Conclusion} \label{conclusion}
% - State your main findings and interpretations
% - Try as far as possible to present perspectives for future work
% - Try to discuss the pros and cons of the methods and possible improvements

\clearpage
\bibliographystyle{plain}
\bibliography{project2.bib}
\end{document}
